\documentclass[12pt]{article}

\usepackage{sbc-template}

\usepackage[utf8]{inputenc}

\usepackage{graphicx}
\usepackage{float}
\usepackage{soul}
\usepackage{color}
\newcommand{\hilight}[1]{\colorbox{yellow}{#1}}

\graphicspath{{\main/imagens/}{imagens/}}

\sloppy

\title{
    Desenvolvimento de documentação e implementação de sistema de informação para
    contribuir na formação do perfil de analista de sistemas
}

\author{Vinícius B. Bruscagini\inst{1}, Carlos R. S. Júnior\inst{1}}

\address{
    Instituto Federal de Educação, Ciência e Tecnologia do Estado de São Paulo\\
    Campus Hortolândia (IFSP-HTO)
    \email{vinicius.bruscagini@aluno.ifsp.edu.br, carlos.rsantos@ifsp.edu.br}
}

\begin{document}

\maketitle

\begin{abstract}
  Regarding difficulties in universities for adapting emerging technologies into curriculum, this paper
  has the objective to create a technical software documentation based in a developed software that
  uses consolidated market technologies. Containing simple and comprehensive language to contribute
  with the formation of IT students.
\end{abstract}

\begin{resumo}
  Com a dificuldade de adequação das instituições de ensino em relação a novas tecnologias, este trabalho possui como principal objetivo
  disponibilizar uma documentação técnica de um \textit{software} desenvolvido com tecnologias
  consolidadas no mercado. Com uma linguagem simples e compreensível para colaborar na formação
  de alunos dos cursos de Análise e Desenvolvimento de Sistemas.
\end{resumo}

\section{Introdução}

A área de TI (Tecnologia da Informação), é uma área que evoluiu muito nas ultimas
décadas~\cite{Pacheco10} e continua evoluindo rapidamente. Novas Tecnologias e ferramentas
surgem em rítmo acelerado e muitas delas ganham popularidade, demandando assim, novas
habilidades para os profissionais de TI\@. Com essa evolução, há uma dificuldade por parte das
instituições de ensino para ofertar conteúdos sobre essas habilidades para estudantes, fazendo com que
muitos desses busquem cursos ou certificações para aprimorar seus conhecimentos.~\cite{macedo09}.
Além disso, de acordo com uma pesquisa realizada por~\cite{wei08} os alunos de cursos de como Engenharia de \textit{Software}
se preocupam em possuir um aprendizado voltado para prática e sempre ter convivência com conceitos
e métodos importantes para o mercado de trabalho.

Baseando-se nessas questões, a produção desse trabalho tem a proposta de desenvolver
um sistema com tecnologias e ferramentas populares que possuem demanda por profissionais,
com o objetivo principal de construir uma documentação
que abrange todos os aspectos desse sistema e que explique com clareza como o software
funciona, quais tecnologias foram usadas e como se deu o seu desenvolvimento. Assim ajudando estudantes a
entenderem alguns conceitos de desenvolvimento de software e como se dão suas aplicações na prática.

% TODO: Falar do Desenvolvimento

Esse artigo possui a seguinte organização: A seção~\ref{FundamentacaoTeorica} nomeada Fundamentação Teórica,
evidencia em subseções os conceitos empregados durante o desenvolvimento deste trabalho. Na sequencia,
a seção~\ref{Metodologia} nomeada Metodologia, descreve o processo realizado para esse desenvolvimento.
A seção~\ref{Desenvolvimento}, Desenvolvimento do Trabalho, mostra as etapas de desenvolvimento para chegar ao objetivo do trabalho, divididos em subseções.
A seção~\ref{Resultados} de Resultados, mostra o que foi implementado e documentado de acordo com o objetivo do trabalho.
Por último a seção~\ref{Conclusao}, Conclusão, resume todo o artigo e mostra possíveis trabalhos futuros.

\section{Fundamentação Teórica}\label{FundamentacaoTeorica}

Nessa seção são apresentados conceitos relacionados ao desenvolvimento de \textit{software} e gerenciamento
de projetos. Esses conceitos foram empregados para realizar o desenvolvimento do \textit{software} conforme descrito na seção anterior.

\subsection{Framework}

Um conceito importante no desenvolvimento de sistemas é o \textit{framework}, de acordo com~\cite{host12} um
\textit{framework} é como um template que conta com diversas funcionalidades. Ele possui o objetivo principal de
resolver problemas recorrentes no desenvolvimento e acelerar esse processo. O
Projeto Pedagógico do Curso de Análise e Desenvolvimento de Sistemas do IFSP - Campus Hortolândia
contém uma disciplina que ensina o uso de um framework para o desenvolvimento. No entanto, a bibliografia básica
que aborda esse \textit{framework} é do ano de 2010 e faz uso da tecnologia JSF (\textit{Java Server Pages}), que hoje em dia é muito pouco usada.~\cite{webtechsurveyJSF}

\subsection{WebService}\label{WebService}

Um \textit{WebService} é um serviço que é oferecido e é por onde aplicações se comunicam com o servidor.
WebServices fazem parte da arquitetura orientada a serviços (SOA).

Um tipo de \textit{WebService} muito utilizado é o REST, que significa
\emph{Representational State Transfer} ele funciona servindo requisições de clientes
para certos serviços. Por exemplo, suponha que um \textit{WebService} REST tenha uma \textit{URL} que fornece
informações de funcionários

\begin{verbatim}
  GET http://meuwebservice/funcionarios
\end{verbatim}

\textit{GET} é o método HTTP usado para requisitar as informações que queremos, então se enviarmos uma requisição
para a URL com o método esperado, o \textit{WebService} irá responder a requisição com dados em formato \textit{JSON} com as informações que queremos, por exemplo:

\begin{verbatim}
  [
    {
      "nome": "Aline",
      "departamento": "Engenharia"
    },
    {
      "nome": "João",
      "departamento": "Finanças"
    }
  ]
\end{verbatim}

Um \textit{WebService} REST pode ter vários URLs, também chamadas de \textit{endpoints}, além de obter informações, podemos realizar
diferentes operações em diferentes \textit{endpoints}, como inserir um objeto no sistema, por exemplo.

\subsection{Qualidade de Software}

A definição de qualidade no contexto da computação nem sempre é um consenso,
existem diferentes terminologias, as quais podem causar problemas para pessoas que não
possuem conhecimento sobre.~\cite{Duarte03}

Um software pode ser considerado de qualidade quando ele atende a todas as necessidades, explicitas
e implícitas para qual ele foi feito.~\cite{Duarte03}

Enquanto ao código, que será o foco do artigo, existem padrões que devem ser seguidos para um código
possuir qualidade. Para isso devemos pontuar que para um código ter qualidade ele deve funcionar, e principalmente,
ser légivel.

\begin{quote}
  ``Muitos programadores iniciantes acham que um bom código é aquele que é correto, ou seja, faz o que tem
  que fazer.''~\cite{Levy04}.
\end{quote}

Um código, para funcionar deve possuir ao menos duas características, ele deve ser \textbf{Correto} e \textbf{Eficiente},
esses pontos são o que fazem um código funcionar. Porém isso não é o suficiente para ser considerado
um código de qualidade. No mundo real, códigos são criados, alterados e revistos, geralmente por diferentes pessoas,
por isso existem duas propriedades adicionais que um código deve possuir, ele deve ser \textbf{Elegante} e \textbf{Testável}
como descritos por~\cite{Levy04}.

Um código elegante e testável dá qualidade ao nosso código pois isso facilita, e muito, sua manutenção.

\subsection{Controle de Versão}

Controle de versão é a prática de rastrear e gerenciar mudanças no código do software~\cite{attlasianGit}.
Essas ferramentas gerenciam o código fonte de projetos ao longo do tempo, e possuem informações detalhadas
de todas as modificações que foram realizadas em todos os arquivos do código fonte.
Alguns dos benefícios que essas ferramentas proporcionam são rastreabilidade (quem fez o quê),
ajudam na resolução de conflitos e aceleram o desenvolvimento. A ferramenta mais famosa para controle de
versão atualmente é o \textit{Git}.

\subsubsection{Git}

O \textit{Git} é a ferramenta mais usada para controle de versão. De acordo com seu website, ele
funciona com o conceito de \textit{commits} e \textit{branches}, em que um \textit{commit}
é uma alteração realizada no código, e uma \textit{branch} é uma ramificação do código, elas são
exemplificadas na Figura~\ref{fig:git-branches}.

\begin{figure}[H]
  \centering
  \includegraphics[width=1\textwidth]{git/git-branches.png}
  \caption{Exemplos de branches em um repositório Git}\label{fig:git-branches}
\end{figure}

A Figura~\ref{fig:git-branches} mostra três \textit{branches} de um repositório,
uma chamada \textit{master}, outra de \textit{Feature 1} e uma outra chamada \textit{Feature 2}.

Durante o desenvolvimento de um \textit{software}, essas ramificações acontecem frequentemente. Nesse exemplo,
há duas ramificações que foram criadas para desenvolver uma funcionalidade no \textit{software}, também chamado de \textit{feature},
ou seja, a partir de uma branch principal \textit{Master}, foram criadas duas \textit{branches} onde o desenvolvimento
de uma funcionalidade foi feita, e ao final desse desenvolvimento, o código feito nessa \textit{branch} foi
unido junto com a \textit{branch} principal.

Com o \textit{software} instalado em um computador podemos usar o comando \verb|git| para realizarmos operações.
Um típico fluxo de um desenvolvimento usando \textit{Git} é mostrado na Figura~\ref{fig:git-lifecycle}.

\begin{figure}[H]
  \centering
  \includegraphics[width=.8\textwidth]{git/git-lifecycle.png}
  \caption{Típico ciclo de desenvolvimento com Git}\label{fig:git-lifecycle}
\end{figure}

O \textit{Git} funciona em repositórios, que é uma pasta onde estão os arquivos de código de um sistema,
a partir desse repositório, modificações são feitas no código, as quais são adicionadas em um \textit{commit}
e esse \textit{commit} é enviado para um repositório remoto, onde outras pessoas da equipe podem obter as modificações feitas.

\subsection{Documentação de Software}

De acordo com~\cite{Forward02softwaredocumentation}, a documentação de um \textit{software} é qualquer
artefato que possui como finalidade informar sobre ele, em qualquer um de seus aspectos.

~\cite{Coelho_2009} nos mostra dois tipos de documentações, são esses tipos

\begin{itemize}
	\item Documentação Técnica
	\item Documentação de Uso
\end{itemize}

As documentações técnicas de um sistema é toda aquela documentação voltada ao desenvolvedor e é
ela que informa como o sistema funciona internamente, sua arquitetura, tecnologias usadas e outros detalhes de implementação.

As documentações de uso são documentos que são focados nos usuários finais do sistema e as vezes também para administratores do sistema,
essas documentações geralmente mostram como usar o sistema.

Ambos os tipos de documentações são muito importantes no processo de desenvolvimento e entrega de softwares. A falta ou
baixo nível de qualidade de documentações atrapalham na compreensão do sistema e podem apresentar
riscos para sua manutenção~\cite{deinvestigaccao}.

\subsection{Metodologias Ágeis}

As metodologias ágeis surgiram na década de 80 para melhorar a área de desenvolvimento de software,
que era algo muito rigoroso naquela época, muitos projetos possuiam problemas e as metodologias ágeis foram então
criadas para tentar solucioná-los.~\cite{Santos05}

\subsubsection{Kanban}

O \textit{Kanban} é uma metodologia que possui cinco princípios de acordo com~\cite{Agile06}
\begin{itemize}
  \item Visualização
  \item Limitar o desenvolvimento em progresso
  \item Medir e gerenciar a sequencia de trabalho
  \item Explicitar o processo
  \item Reconhecimento de oportunidades
\end{itemize}

O \textit{Kanban} funciona com quadros, onde em cada quadro temos várias seções.
Nessas seções podemos agrupar tarefas.

Em um quadro Kanban podemos agrupar tarefas, definir datas e responsáveis, entre vários outros
detalhes, o que facilita a organização e visão das tarefas do projeto.

Existem várias ferramentas que implementam essa metodologia.
Uma delas é o \emph{GitHub Projects}, que usaremos neste trabalho. É usada para gerenciamento de projetos
com a metodologia \textit{Kanban}. A interface desse programa é mostrada na Figura~\ref{fig:github-board}.

\begin{figure}[H]
  \centering
  \includegraphics[width=1\textwidth]{github/github-board.png}
  \caption{Tela de um quadro Kanban do \emph{GitHub Projects}}\label{fig:github-board}
\end{figure}


% TODO: Falar de um pouco de Testes unitários aqui

\section{Metodologia}\label{Metodologia}

Para início do desenvolvimento foi realizada a definição das tecnologias do sistema, para isso, foram feitas
pesquisas bibliográficas em \textit{WebSites} como \textit{GitHub} e \textit{Stackoverflow}. Estes \textit{WebSites}
são famosos na área de desenvolvimento, eles forneceram informações importantes sobre quais
são as tecnologias mais ativas, mais comentadas e outros dados que foram usados para a escolha das tecnologias
usadas no desenvolvimento do sistema.

Com as tecnologias definidas, o próximo passo foi a análise e documentação de requisitos do sistema,
e com esses requisitos, foram criados diagramas de classes e casos de uso.

Para o desenvolvimento, foram usados recursos oferecidos pela plataforma \textit{GitHub}, esses recursos são:

\begin{itemize}
  \item Repositório \textit{Git}
  \item Quadro \textit{Kanban} para gerenciamento de projetos
  \item \textit{Wiki} onde será criada a documentação técnica do sistema
\end{itemize}

\section{Desenvolvimento do Trabalho}\label{Desenvolvimento}

Em Maio de 2021 foi realizada pelo site~\cite{stack11}, uma pesquisa com desenvolvedores de todo o mundo para
coletar dados geográficos, sociais e de uso de tecnologias de seus usuários. Esses dados nos permitem ter
informações importantes sobre o uso de tecnologias em geral. A pesquisa nos dá três conjuntos
de informações relevantes para usarmos no trabalho

\begin{itemize}
  \item Frameworks para Web mais utilizados
  \item Frameworks gerais mais utilizados
  \item Ferramentas para deploy e gerenciamento mais usadas
  \item Banco de Dados mais utilizados
\end{itemize}

Os resultados dessas categorias, entre desenvolvedores profissionais, são exibidos nas
Figuras~\ref{fig:web-frameworks},~\ref{fig:general-frameworks},~\ref{fig:tools} e~\ref{fig:databases}, respectivamente.

\begin{figure}[H]
  \centering
  \includegraphics[width=1\textwidth]{stackoverflow/web_frameworks_usage.png}
  \caption{Gráfico mostrando os frameworks Web mais usados}\label{fig:web-frameworks}
\end{figure}

\begin{figure}[H]
  \centering
  \includegraphics[width=1\textwidth]{stackoverflow/general_framework_usage.png}
  \caption{Gráfico mostrando os frameworks gerais mais usados}\label{fig:general-frameworks}
\end{figure}

\begin{figure}[H]
  \centering
  \includegraphics[width=1\textwidth]{stackoverflow/used_tools.png}
  \caption{Gráfico mostrando as ferramentas mais usadas}\label{fig:tools}
\end{figure}

\begin{figure}[H]
  \centering
  \includegraphics[width=1\textwidth]{stackoverflow/databases.png}
  \caption{Gráfico mostrando os SGBDs mais usados}\label{fig:databases}
\end{figure}

Baseando-se nesses dados de utilização e na experiência do autor em certas tecnologias e ferramentas,
foi optado o uso das tecnologias e arquitetura apresentadas nas próximas seções.

\subsection{Arquitetura e Tecnologias}

A arquitetura escolhida para o sistema foi um \textit{WebService REST}, como mostrado na seção~\ref{WebService}.
Além do \textit{WebService} foi decidido a criação de uma aplicação cliente que acesse o \textit{WebService}.

As tecnologias que foram usadas para a implementação dessas aplicações são apresentadas nas próximas seções.

\subsubsection{ASP.NET Core}

O \textit{ASP.NET Core} faz parte da família de tecnologias \textit{.NET} que são desenvolvidas pela \textit{Microsoft}.
Ele nos permite criar qualquer tipo de aplicação \textit{web}.

A Plataforma \textit{.NET} possui uma longa história, por grande parte dessa história, suas
tecnologias foram exclusivas para o sistema operacional \textit{Windows}. Em 2014 a \textit{Microsoft} recriou
essa plataforma do zero, hoje toda a plataforma é \textit{open-source} e funciona em qualquer
sistema operacional.

\subsubsection{Entity Framework Core}

O \textit{Entity Framework Core} é um \textit{framework ORM} que é usado em conjunto com o \textit{.NET}
como explicitado por~\cite{ormHibernate}:

\begin{quote}
  `Um \textit{framework ORM} provê uma metodologia e mecanismo
  para sistemas orientados a objetos manterem seus dados em um banco de dados de maneira
  segura, com controle de transações, e tudo isso expressado em código orientado a objeto'.
\end{quote}

De maneira prática, um \textit{ORM} permite manipularmos entidades em um banco de dados
sem precisarmos mexer com \textit{SQL}. Eles geralmente funcionam mapeando as classes
que criamos em nosso código, e com essas informações ele cria o banco de dados já com os relacionamentos
entre as tabelas e tudo mais o que definimos nas classes.

O \textit{Entity Framework} é o \textit{ORM} oficial do \textit{.NET}, possui amplo suporte e uso. Ele será
usado nesse projeto para facilitar a manipulação dos dados no banco de dados.

\subsubsection{Vue.js}

O \textit{Vue} é um \textit{framework} para desenvolvimento de aplicações \textit{web}, usando a linguagem \textit{JavaScript},
ele nos permite o desenvolvimento de aplicações reativas e possibilita o reuso
de código através de componentes. Abaixo um pequeno código \textit{HTML} de um arquivo vue

\begin{verbatim}
  <div>
    <p v-if="seen">Agora você me viu</p>
  </div>
\end{verbatim}

Esse trecho de código exibe o texto `Agora você me viu' caso a variável \verb|seen| seja verdadeira.

O \textit{Vue} nos dá a opção de usar a linguagem \textit{TypeScript} ao invês do \textit{JavaScript}, ele possui
várias bibliotecas que melhoram a legibilidade e arquitetura de um componente vue.
Abaixo um código de um componente \textit{Vue} usando \textit{TypeScript}

\begin{verbatim}
<template>
  <p>Texto</p>
</template>

<script lang="ts">
import Vue from "vue";
import Component from "vue-class-component";

@Component
export default class About extends Vue {

}
</script>
\end{verbatim}

O \textit{Vue} possui um ecosistema de bibliotecas adicionais para ajudar no desenvolvimento de aplicações.
Uma dessas bibliotecas é o \textit{Vuetify}, ele fornece para o desenvolvedor componentes para interface
como botões, cards, campos de textos, etc.

\subsection{Definição de Requisitos}

Com o processo e as tecnologias definidas. Agora será decidido o domínio do sistema a ser desenvolvido
e seus requisitos.

Como o foco do trabalho não é o domínio da aplicação, foi optado por copiar o funcionamento de uma aplicação
existente, a aplicação escolhida é o \textit{Reddit}.

O \textit{Reddit}, de acordo com seu website é:

\begin{quote}
  `A casa de milhares de comunidades, conversas sem
  limite e interação humana autentica.'
\end{quote}

O \textit{Reddit} consiste de comunidades que tratam de determinado assunto, nessas comunidades, usuários
podem se inscrever e realizar postagens, as quais outros usuários dessa comunidade virão e podem
dar um voto positivo ou negativo para essa postagem.

Foram criados dois diagramas de casos de uso que representam as funcionalidades esperadas no sistema,
eles são apresentados nas Figuras~\ref{fig:welcome_diagram} e~\ref{fig:home_diagram}.

%TODO Diagrama está com nome antigo [Simple Forum]

\begin{figure}[H]
    \centering
    \includegraphics[width=0.6\textwidth]{diagrams/welcome_diagram.png}
    \caption{Diagrama de caso de uso da tela inicial}\label{fig:welcome_diagram}
\end{figure}

\begin{figure}[H]
    \centering
    \includegraphics[width=0.6\textwidth]{diagrams/home_diagram.png}
    \caption{Diagrama de caso de uso da tela \textit{home}}\label{fig:home_diagram}
\end{figure}

Junto com os casos de uso foram definidos as propiedades das entidades do sistema
a partir da criação de um diagrama de classe, o diagrama resultante é exibido
na Figura~\ref{fig:classes_diagram}.

\begin{figure}[H]
    \centering
    \includegraphics[width=0.8\textwidth]{diagrams/classes_diagram.png}
    \caption{Diagrama de classes do sistema}\label{fig:classes_diagram}
\end{figure}



\subsection{Ambiente de Desenvolvimento}

A ferramenta de desenvolvimento escolhida para implementação
das aplicações foi o \textit{Visual Studio Code} em conjunto com extensões disponíveis
no seu repositório para facilitar seu uso com as tecnologias escolhidas para o projeto. Além
da própria ferramenta de desenvolvimento, foi necessário a instalação dos SDKs
(Kit de Desenvolvimento) das tecnologias que foram utitlizadas. Foram eles o \textit{dotnet} para a implementação do \textit{WebService},
em conjunto com o \textit{Node.js} e \textit{Vue CLI} para implementação do cliente \textit{web}.

Foi utilizado, como auxílio para o desenvolvimento do \textit{WebService} que é uma API REST, o
\textit{software} \textit{Insomnia REST Client} para realização de testes.

Para a criação do banco de dados \textit{MySQL} foi utilizado a ferramenta \textit{Docker} para
seu gerenciamento via contâiner, e para acesso e manipulação desse \textit{MySQL} foi utilizado o software \textit{Dbeaver}.
Além dessas ferramentas, foi utlilizado o site \textit{mockaroo} que oferece scripts para preenchimento
de banco de dados com informações de testes. Facilitando o processo de desenvolvimento.

\subsection{Implementação}

Como mencionado anteriormente, foi utilizado a metodologia \textit{Kanban} para gerenciamento
de tarefas, a implementação dos sistemas foram realizadas baseados nessas tarefas.
Para implementação de algumas dessas tarefas houveram atrasos devido a necessidade de aprendizagem
para a resolução daquele problema técnico. Como por exemplo a implementação de sistema de segurança nas aplicações.

\subsection{Documentação}

Após a finalização da implementação das aplicações, foi construída uma documentação
dos \textit{softwares}, essa documentação foi feita na plataforma \textit{GitHub} na funcionalidade
de \textit{Wiki}. Foi criada uma \textit{Wiki} em cada repositório da aplicação, uma para a aplicação em \textit{Vue,js}
e outra para a API em \textit{.NET}

\section{Resultados}\label{Resultados}

As duas aplicações foram implementadas junto com todas as funcionalidades propostas.
Dois \textit{prints} da aplicação \textit{web} são exibidos nas Figuras~\ref{fig:welcome} e~\ref{fig:home}.

\begin{figure}[H]
    \centering
    \includegraphics[width=1\textwidth]{prints/welcome.png}
    \caption{Tela de bem-vindo}\label{fig:welcome}
\end{figure}

Essa é a primeira tela da aplicação, onde o usuário pode fazer seu \textit{Login} ou se cadastrar.

\begin{figure}[H]
    \centering
    \includegraphics[width=1\textwidth]{prints/home.png}
    \caption{Tela \textit{home}}\label{fig:home}
\end{figure}

Essa tela é a \textit{Home}, é onde o usuário vê as postagens de outros usuários em comunidades.
A partir dessa tela o usuário pode também entrar em comunidades, fazer postagens, ou realizar uma pesquisa.

O Desenvolvimento da API em \textit{.NET} foi finalizadada e todas as funcionalidades
usadas pela aplicação \textit{Web} foram implementadas com restrições de acesso e esquemas de
segurança, um \textit{print} dos endpoints disponíveis nessa \textit{API} é exibido
na Figura~\ref{fig:swagger} através da biblioteca \textit{Swagger}.

\begin{figure}[H]
    \centering
    \includegraphics[width=1\textwidth]{prints/swagger.png}
    \caption{Print da tela do Swagger}\label{fig:swagger}
\end{figure}

Além do desenvolvimento, foi criada a documentação técnica dessas aplicações, elas estão
na \textit{Wiki} dos repositórios \textit{Git}. Além dos arquivos \textit{README} que mostram
uma introdução das aplicações na tela do repositório.

Os repositórios estão localizados nos links especificados a seguir:
\begin{itemize}
	\item https://github.com/V11-0/ForyumServer (WebService)
	\item https://github.com/V11-0/ForyumWeb (Aplicação Web)
\end{itemize}

\section{Conclusão}\label{Conclusao}

O artigo mostrou a dificuldade do ensino em relação as novas tecnologias no desenvolvimento de \textit{software}
e propôs a criação de um sistema com uma documentação técnica para ajudar estudandes sobre alguns dos
conceitos e tecnologias usadas no desenvolvimento de \textit{software}, além de prover uma abordagem prática
sobre o assunto. Foram implementados duas aplicações sendo um \textit{WebService} com tecnologias \textit{.NET} e
uma aplicação \textit{web} com o \textit{framework Vue}. O código fonte e a documentação técnica dessas duas aplicações
estão disponíveis \textit{online} na plataforma \textit{GitHub}.

Como trabalhos futuros, uma pesquisa de opinião poderia ser feita para estudantes de TI em relação ao material
disponibilizado por este trabalho. Além da documentação, outros conceitos que não foram abrangidos nesse trabalho poderiam
ser explorados como questão de \textit{Pipelines} para realização de entrega contínua (\textit{CI/CD}).
Outra recomendação poderia ser o aprimoramento e criação de novos recursos para \textit{software} criado nesse trabalho.

\bibliographystyle{sbc}
\bibliography{bibliografia}

\end{document}
